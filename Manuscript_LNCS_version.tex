% This is LLNCS.DEM the demonstration file of
% the LaTeX macro package from Springer-Verlag
% for Lecture Notes in Computer Science,
% version 2.4 for LaTeX2e as of 16. April 2010
%
\documentclass{llncs}
%
\usepackage{makeidx}  % allows for indexgeneration
%
\begin{document}
%
\frontmatter          % for the preliminaries
%
\pagestyle{headings}  % switches on printing of running heads

\mainmatter              % start of the contributions
%
\title{Collaborative platforms for streamlining workflows in Open Science}
%
\titlerunning{Hamiltonian Mechanics}  % abbreviated title (for running head)
%                                     also used for the TOC unless
%                                     \toctitle is used
%
\author{Konrad U. F\"orstner\inst{1}\inst{2} \and Gregor Hagedorn\inst{3}
Claudia Koltzenburg\inst{4} \and M. Fabiana Kubke\inst{5} \and Daniel Mietchen\inst{6}}
%
\authorrunning{Konrad U. Förstner et al.} % abbreviated author list (for running head)
%
%%%% list of authors for the TOC (use if author list has to be modified)
\tocauthor{Konrad U. Förstner, Gregor Hagedorn, Claudia Koltzenburg, M. Fabiana Kubke, Daniel Mietchen}
%


\institute{Institute for Molecular Infection Biology\\
University of W\"urzburg \\ 
D-97080 W\"urzburg, Germany
\and
Research Centre for Infectious Diseases\\
University of W\"urzburg\\
D-97080 W\"urzburg, Germany
\and
Julius K\"uhn-Institute\\
Federal Research Center for Cultivated Plants\\
Berlin, Germany
\and
Claudia Koltzenburg\\
Managing editor of Cellular Therapy and Transplantation (CTT)\\
\ \\
\and
Department of Anatomy with Radiology,
University of Auckland\\
\
\and
Daniel Mietchen\\
Science 3.0}


\maketitle              % typeset the title of the contribution

\begin{abstract}
Despite the internet's dynamic and collaborative nature, scientists
continue to produce grant proposals, lab notebooks, data files,
conclusions etc. that stay in static formats or are not published
online and therefore not always easily accessible to the interested
public. Because of limited adoption of tools that seamlessly integrate
all aspects of a research project (conception, data generation, data
evaluation, peer-reviewing and publishing of conclusions), much effort
is later spent on reproducing or reformatting individual entities
before they can be repurposed independently or as parts of articles.\\

We propose that workflows - performed both individually and
collaboratively - could potentially become more efficient if all steps
of the research cycle were coherently represented online and the
underlying data were formatted, annotated and licensed for reuse. Such
a system would accelerate the process of taking projects from
conception to publication stages and allow for continuous updating of
the data sets and their interpretation as well as their integration
into other independent projects.\\

A major advantage of such workflows is the increased transparency,
both with respect to the scientific process as to the contribution of
each participant. The latter point is important from a perspective of
motivation, as it enables the allocation of reputation, which creates
incentives for scientists to contribute to projects. Such workflow
platforms offering possibilities to fine-tune the accessibility of
their content could gradually pave the path from the current static
mode of research presentation into a more coherent practice of open
science.\\

\keywords{Open Science, Virtual Research Environments,
  collaboratories, workflow platform, automation, transparency,
  reproducibitliy, reputation, funding, Open Hardware}
\end{abstract}
%
\section{Introduction}
%

Like most areas of today's life, science has dramatically changed
since the advent of the internet. However, the transformation that has
taken place until now is just the tip of the iceberg. In the
following, we want to discuss the mostly underutilized potential of
representing all aspect of science in collaboratively used online
workflow platforms. Since such platforms could help to realize Open
Science, transparency of the funding cycles and access to all data in
the research process, we will shed light on this special aspect and
make recommendations regarding implementations.\\

While there are numerous projects developing and applying so called
Virtual Research Environments (VRE) - also known as Collaboratories -
covering selected stages of the scientific process, a platform
spanning every phase is missing so far \cite{Carusi}. Technically
overcoming such gaps and creating a seamless transition from bench to
publication could speed up the research and, with it, the generation,
distribution and reuse of knowledge.

\section{The scientific workflow in open VREs}

\subsection{Conception and project planning}

Independent of the nature of a research endeavor - hypotheses-driven
or data-driven, performed by a single person or a team - a solid
conception phase is the crucial basis for every project. Despite
today's common practice of limiting this phase to a small group of
people, utilizing collective intelligence during the conception phase
could help to avoid redundant research and to improve the design of
the study. As the complexity and scope of scientific projects are
increasing, the application of project management tools can be useful
for managing the processes and parties involved.

\subsection{Experiments and data generation}

Today, data generation in academic research continues to rely strongly
on manual labor. While this is mostly due to the relative low cost of
labor force resulting from the academic system and the limited
interdisciplinary education of science and engineering, the high
potential of automation is mostly neglected. Not only could the
efficiency of invested labor be improved by automation, but also
reproducibility could be significantly increased. To make this
affordable for the broader research community, a shift from siloed
proprietary devices to well-documented pieces of standardized,
open-source hardware developed by the scientific community itself in
cooperation with potential vendors is needed. Open hardware platforms
like Arduino \cite{Arduino} could offer starting points for such a
development and first example of such tools are available
(e.g. OpenPCR \cite{OpenPCR}). The devices could and should enrich the
primary data with further metadata, convert them into semantified
formats and directly upload the output into online repositories.\\

One promising example which visualizes the potential of such
automation of otherwise quite labor-intensive research is the robot
scientist ADAM \cite{King}. The streamlining of mechanical steps and
the evaluation of results would benefit from formal languages that
describe the necessary procedures and make the design and exchange of
experimental setups easy \cite{Soldatova}. As a long term goal,
scientists would mostly engage in programming experiments and
engineering the system to automate those steps that have been
performed manually so far. The motto ``work on the system, not in the
system'' should guide this development.

\subsection{Data release}

The online release of experimentally generated data should be done
shortly after the generation and can potentially happen in real
time. Downstream analysis within the research project but also the
reuse by other parties should be kept in mind when selecting data
formats. These should, as far as possible, be non-proprietary, machine
readable (semantically enriched) and common for the respective domain
of research. If no format fulfills all these requirements, the
conversion into alternative formats should be permitted. Access to the
data could take place via a web interface or domain specific
clients. Especially for large or highly accessed data sets, the
additional distribution via peer-to-peer networks is recommended.

\subsection{Data analysis}

Since every step in the data analysis should be transparent and easily
reproducible, it should take place preferably in the proposed
platform, too. Systems like the analysis workflow tool Taverna
\cite{Hull} could be used for such processing. Already today, many
research institution offer grid computing infrastructure for such
purposes. Analyses using external tools, especially GUI-tools that do
not offer any possibility to log the performed actions, should be
avoided if possible, as otherwise documentation has to be created
manually. For some computationally intensive analyses the use of
shared systems is a more economical usage of the needed
infrastructure, provided the management overhead does not exceed the
computational efficiency gain. As done for the raw experimental data,
the protocols and the result of the data processing should be
documented and stored in repositories to be accessible.

\subsection{Knowledge generation}

The results of analytical processing as well as the raw data can be
used by scientists - or machines \cite{Schmidt} - to draw conclusions
and to generate knowledge out of the available information in a well
documented way. The platform should assist to make this happen
collaboratively by offering commenting and rating of
statements. Discussions - text, audio- and/or video-based - should be
recorded to make the path to finding reconstructible.

\subsection{Final publication}

As documentation of every step is an inherent feature of the workflow,
the final publications resulting from a study can be short reports
linking to the major outcomes and putting them into the scientific
context. The platform should offer functionalities to perform open
peer-review of this final report.

\section{Implementation}

\subsection{Technology}

As shown above, the many building blocks of a complete scientific
workflow already exist and only need to be connected seamlessly. The
development of open standards defining the required interfaces of
these parts could enable different parties to assemble the pieces into
a consistent workflow and to add further needed parts. This would
offer the possibility to implement a platform either as one monolithic
application or as separate interacting and exchangeable units.

\subsection{Funding}

Of similar importance as the technical realization is the adaptation
of scientific culture and funding policies. While research
institutions like the National Institutes of Health (US) or the
Welcome Trust (UK) already require open access for final peer-review
manuscripts that results from research they funded
\cite{NIH,WelcomeTrust}, the regulations are much weaker for the
underlying data, and almost nonexistent for proper
annotation. However, the first attempts to establish such requirements
are on the horizon \cite{Vickers}.

\subsection{Licensing}

The default copyright restrictions in most jurisdictions hamper the
reuse of data. It is therefore highly desirable that, with very few
exceptions, each entity generated in the research process is
explicitly published under a less restrictive license, e.g., the ones
offered by Creative Commons \cite{CC} or is released into the public
domain. As the latter concept may differ or be missing in some
countries, release through the CC0 license \cite {CC0} is
recommended.

\subsection{Reputation}

The gain of reputation is the most important incentive for
scientists. It is currently mostly determined on the basis of
publications in scientific journals and the related measure of success
in funding applications. As every contribution to a research project
can be attributed to a distinct person and could be rated by others,
the allocation of reputation is an inherent element of the proposed
platform. The connection to research identifiers like ORCID
\cite{ORCID} and the analysis of such microcontributions could
assemble a precise image of a scientist's skills and achievements.

\section{Challenges}

As stated above, considering the allocation of reputation and funding
in science is crucial when redesigning scientific processes. To bridge
a transient phase until the suggested political changes have taken
place, fine granular access control in the research workflow platform
could permit that the technology is adapted by scientists despite
objection regarding the loss of reputation. With such a control in
place, the full process could be opened up after the final publication
or at any other desired time.\\

It is very unlikely that there will be one single platform that can
fulfill the requirements of all scientific domains. Building and
maintaining completely independent platforms for each domains, on the
other hand, may not be sustainable. A modular and flexible system,
where possible re-using industry standard software is therefore called
for.\\

Projects presently exploring this are, e.g.:

\begin{itemize}
  \item the eSciDoc platform which builds on the open-source
    repository software Fedora Commons \cite{Feudora} and is mainly
    developed for the for the Max Planck Society has a similar aim and
    strategy \cite{Dreyer}.
  \item the FP7 funded Virtual Biodiversity Research and Access
    Network for Taxonomy'' (ViBRANT) \cite{vbrant,Roberts,Blagoderov}
\end{itemize}

They span data collection, analysis and publishing (in collaboration
with Pensoft Publishers), are based on the established open-source
platforms Drupal \cite{Drupal} and Mediawiki \cite{MediaWiki} and
equipped with specific extensions.

\paragraph{Notes and Comments.}

%
% ---- Bibliography ----
%
\begin{thebibliography}{5}
%
\bibitem{Carusi} Annamaria Carusi, Torsten Reimer. Virtual Research
  Environment - Collaborative Landscape Study. JISC. pp. 72-24 2010.

\bibitem{Arduino} Arduino, http://www.arduino.cc/

\bibitem{OpenPCR} OpenPRC, http://openpcr.org/

\bibitem{King} Ross D. King, Jem Rowland, Stephen G. Oliver, Michael
  Young, Wayne Aubrey, Emma Byrne, Maria Liakata, Magdalena Markham,
  Pinar Pir, Larisa N. Soldatova, Andrew Sparkes, Kenneth E. Whelan,
  Amanda Clare. The automation of science. Science. 3 April 2009:
  Vol. 324 no. 5923 pp. 85-89

\bibitem{Soldatova} Larisa N Soldatova, Ross D King. An ontology of
  scientific experiments. J R Soc Interface. 2006 Dec 22;3(11):795-803

\bibitem{Hull} Duncan Hull, Katy Wolstencroft, Robert Stevens, Carole
  Goble, Mathew R. Pocock, Peter Li, Tom Oinn. Taverna: a tool for
  building and running workflows of services Nucleic Acids Res. 2006
  Jul 1;34(Web Server issue):W729-32.

\bibitem{Schmidt} Michael Schmidt and Hod Lipson. Distilling free-form
  natural laws from experimental data.  Science. 2009 Apr
  3;324(5923):81-5

\bibitem{NIH} NIH Public Access Policy Details,
  http://publicaccess.nih.gov/policy.htm

\bibitem{WelcomeTrust} Welcome Trust Open access policy,
  http://www.wellcome.ac.uk/About-us/Policy/Policy-and-position-statements/WTD002766.htm

\bibitem{Vickers} Andrew J Vickers. Making raw data more widely
  available. BMJ. 2011; 342:d2323

\bibitem{CC} Creative Commons, http://creativecommons.org/

\bibitem{CC0} CC0, http://creativecommons.org/publicdomain/zero/1.0/

\bibitem{ORCID} ORCID, http://orcid.org/

\bibitem{Feudora} Feudora Commons, http://fedora-commons.org/

\bibitem{Dreyer} Malte Dreyer, Ulla Tschida, Natasa Bulatovic, Matthias
  Razum. eSciDoc - a Scholarly Information and Communication Platform
  for the Max Planck Society. German e-Science Conference,
  Baden-Baden. 2007

\bibitem{vbrant} ViBRANT, http://vbrant.eu

\bibitem{Roberts} Dave Roberts, Vince Smith. ViBRANT - Virtual
  Biodiversity Research and Access Network for Taxonomy. Tools for
  identifying biodiversity: progress and problems. Proceedings of the
  International Congress, Paris. September 20-22, 2010. Edited by Pier
  Luigi Nimis and R ́gine Vignes Lebbe. p. 54

\bibitem{Blagoderov} Vladimir Blagoderov, Irina Brake, Teodor
  Georgiev, Lyubomir Penev, David Roberts, Simon Ryrcroft, Ben Scott,
  Donat Agosti, Terry Catapano, Vincent S. Smith. Streamlining
  taxonomic publication: a working example with Scratchpads and
  ZooKeys. Zookeys. 2010; 50: 17–28

\bibitem{Drupal} Drupal, http://drupal.org

\bibitem{MediaWiki} MediaWiki, http://www.mediawiki.org


\end{thebibliography}

\end{document}
